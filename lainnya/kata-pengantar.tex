\begin{center}
  \Large
  \textbf{KATA PENGANTAR}
\end{center}

\addcontentsline{toc}{chapter}{KATA PENGANTAR}

\vspace{2ex}

% Ubah paragraf-paragraf berikut dengan isi dari kata pengantar

Segala puji dan syukur penulis panjakat kepada Allah Swt,
karena berkat rahmat dan hidayah-Nya, penulis dapat menyelesaikan
penelitian ini dengan judul \textbf{"Implementasi Deep Reinforcement Learning pada Hexagonal Grid Turn-Based Strategy Game"}.
Penelitian ini diajukan sebagai salah satu syarat untuk memperoleh gelar Sarjana Teknik
dari Departemen Teknik Komputer, Fakultas Teknologi Elektro dan Informatika Cerdas, Institut
Teknologi Sepuluh Nopember, Surabaya. Selama proses penelitian dan penulisan, penulis
banyak menerima bantuan dan dukungan dari berbagai pihak. Untuk itu, penulis ingin menyampaikan
ucapan terima kasih yang sebesar - besarnya kepada:

\begin{enumerate}[nolistsep]

  \item Bapak dan Ibu, serta saudara - saudari penulis yang selalu memberikan doa dan dukungan selama masa studi.

  \item Bapak Dr.Supeno Mardi Susiki Nugroho ST., M.T selaku Kepala Departemen Teknik Komputer Fakultas Teknologi Elektro dan Informatika Cerdas, Institut Teknologi Sepuluh Nopember Surabaya dan selaku dosen pembimbing I yang telah bersedia membimbing, membantu, dan memberikan dukungan dalam proses penyelesaian penelitian.
  \item Bapak Mochamad Hariadi, ST., M.Sc., Ph.D. selaku dosen pembimbing II yang telah bersedia membimbing, membantu, dan memberikan dukungan dalam proses penyelesaian penelitian.
  \item Bapak/Ibu dosen pengajar, pengurus tata usaha, karyawan Departemen Teknik Komputer Fakultas Teknologi Elektro dan Informatika Cerdas Institut Teknologi Sepuluh Nopember.
  \item Rekan - rekan angkatan e58, Teknik Komputer dan Laboratorium B201 Telematika Departemen Teknik Komputer ITS.
  \item Berbagai pihak yang terlibat selama proses penelitian dan penulisan skripsi dari awal hingga akhir yang tidak bisa penulis sebutkan namanya satu - persatu.

\end{enumerate}

Penulis menyadari bahwa masih terdapat kekurangan dalam penulisan penilitan ini. Atas segala kekurangan, penulis sangat
terbuka untuk menerima kritik dan saran yang bersifat membangun dari berbagai pihak agar dapat menyempurnakan penelitian ini. 
Akhir kata, semoga penelitian ini dapat bermanfaat bagi para peneliti, almamater, dan negara di masa mendatang.

\begin{flushright}
  \begin{tabular}[b]{c}
    % Ubah kalimat berikut dengan tempat, bulan, dan tahun penulisan
    Surabaya, November 2022\\
    \\
    \\
    \\
    \\
    % Ubah kalimat berikut dengan nama mahasiswa
    Dafa Fidini Asqav
  \end{tabular}
\end{flushright}
