\begin{center}
  \large\textbf{ABSTRAK}
\end{center}

\addcontentsline{toc}{chapter}{ABSTRAK}

\vspace{2ex}

\begingroup
  % Menghilangkan padding
  \setlength{\tabcolsep}{0pt}

  \noindent
  \begin{tabularx}{\textwidth}{l >{\centering}m{2em} X}
    % Ubah kalimat berikut dengan nama mahasiswa
    Nama Mahasiswa    &:& Dafa Fidini Asqav \\

    % Ubah kalimat berikut dengan judul tugas akhir
    Judul Tugas Akhir &:&	Implementasi Deep Reinforcement Learning pada Hexagonal Grid Turn-Based Strategy Game \\

    % Ubah kalimat-kalimat berikut dengan nama-nama dosen pembimbing
    Pembimbing        &:& 1. Dr. Supeno Mardi Susiki Nugroho, S.T., M.T \\
                      & & 2. Mochamad Hariadi, ST., M.Sc., Ph.D. \\
  \end{tabularx}
\endgroup

% Ubah paragraf berikut dengan abstrak dari tugas akhir
Game strategi merupakan permainan di mana pemain mengambil keputusan strategis di dalam game untuk menyelesaikan tujuan. 
Salah satu game tersebut merupakan \emph{Civilization VI} (Civ6).
Dalam game ini, pemain melakukan aksi bergilir dengan lawannya dalam area map yang tersusun dari lantai segi enam (\emph{hexagonal grid}).
Civ6 merupakan game dengan aspek 4X (\emph{Exploration, exploitation, expansion, extermination}). Aspek aspek 4X tersebut menyebabkan game ini menjadi kompleks.
Hal ini menjadi tantangan bagi \emph{game developer} untuk membuat lawan AI yang dapat memberikan tantangan yang cukup terhadap pemain. 
Akan tetapi, game strategi seperti Civ6 masih memiliki kapabilitas agen berbasis AI yang belum optimal.
Berkembangnya bidang \emph{Deep Reinforcement Learning} (DRL) menawarkan teknologi AI yang belum memungkinkan sebelumnya.
Dalam penelitian ini, dirancanglah sebuah \emph{environment} yang mengikuti mekanisme \emph{combat} dalam Civ6
sebagai media implementasi DRL. Terdapat dua agen dalam \emph{environment} ini: agen \emph{attacker} dan \emph{defender}.
Kedua agen memiliki tujuan yang berbeda (\emph{asymmetrical}) dan saling berlawanan (\emph{adversarial}).
Terdapat empat algoritma \emph{state of the art} (SOTA) yang digunakan dalam eksperimen penelitian ini:
DQN, APE-X DQN, PPO, dan IMPALA. Dari hasil experimen, didapatkan bahwa APE-X DQN memiliki performa terbaik bagi agen \emph{attacker}
dan agen \emph{defender}. Agen \emph{attacker} APE-X DQN mampu menghancurkan kota secara konsisten sebelum 2 juta \emph{environment steps}.
Akan tetapi, APE-X DQN menggunakan CPU dan RAM lebih banyak dari algoritma lain, dengan penggunaan CPU sebanyak 79.95\% dan RAM sebanyak 82.3\%.

% Ubah kata-kata berikut dengan kata kunci dari tugas akhir
Kata Kunci: Game strategi, \emph{Deep Reinforcement Learning, Artificial Intelligence, Machine Learning, Civilization VI, DQN, APE-X, PPO, IMPALA}
